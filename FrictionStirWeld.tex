\documentclass[]{article}
 \usepackage{amsmath}
 \usepackage{amssymb}
% \usepackage{amsthm}
% \usepackage{textcomp}
 \parindent=0mm
 \parskip=2.5mm
% \parindent=0mm
% \renewcommand{\arraystretch}{1.35}
% \usepackage{cite}

%\usepackage[acronym,nonumberlist]{glossaries}
%\makeglossaries 
\usepackage{graphicx}
\graphicspath{{Figures/}}
\usepackage[small,bf,up]{caption}
\providecommand{\diff}[3]{\frac{d^{#3} #1}{d #2^{#3}}}
\providecommand{\pdiff}[3]{\frac{\partial^{#3} #1}{\partial #2^{#3}}}
\providecommand{\abs}[1]{\left \lvert#1\right \rvert}
\providecommand{\etal}[0]{\textit{et al. }}
\oddsidemargin 15mm                        % 1 inch + lefthand margin on odd numbered pages
\evensidemargin 15mm                       % 1 inch + lefthand margin on even numbered pages
\textwidth 145mm  

%\usepackage[round]{natbib}
%

\usepackage[square,numbers,comma,sort&compress]{natbib}
\usepackage{listings} 
%\bibliographystyle{plainnat}
\bibliographystyle{unsrt}
%\usepackage{nomencl}
%\nomlabelwidth=15mm
%\makeindex
%\makenomenclature

\date{\today}
\title{In-Situ Repair of PE pipe by Friction Stir Welding}
\author{Richard Collins}



\begin{document}
\maketitle
\section{Vision}


\section{Problem}

There exists clear evidence that there is significant leakage in MDPE mains laid 
by the UK water industry. In 2010 UKWIR 1 reported electrofusion jointing as 
the predominant cause of joint failure on PE mains with a calculated failure 
rate of between 3 and 4 failures per 100km per year. Although there is a lack 
of more recent data there is a widely held view within the industry that 
failure rates on PE mains have not reduced in the past decade and there is still
a significant problem.


\section{Proposed Solution}

\subsection{Friction Stir Welding}
\subsubsection{Friction Stir Welding of PE}



\end{document}
